\section{Mobile Platform Security}

\subsection{History}
How is mobile security evolution different from PC security?

\paragraph{Different Shareholders}
\begin{itemize}
    \item Mobile network operators
    \item End users
    \item Regulators
\end{itemize}{}

\subsection{Mobile OS Security}
\paragraph{Security-related design decisions}
\begin{itemize}
    \item Software distribution
    \item Application isolation
    \item Defining access to resources
    \item Sharing functionality
\end{itemize}{}

\paragraph{Android architecture}
Linux-base OS, open source, many variants\\

\paragraph{Software Distribution}
\begin{itemize}
    \item Android allows software from multiple sources (unlike iOS).
    \item Marketplace provides security scanning.
    \item Application signing: Developer self-signs app (public key identifies developer, but a certain key does not imply developer @ UBS) 
\end{itemize}{}

\paragraph{Application isolation}
\begin{itemize}
    \item Normal process separation
    \item Additionally Seperate UIDs for all applications
    \item Binder mediates IPC (reference monitor)
    \item Applications run in separate vm's
\end{itemize}{}

\paragraph{Defining Access to resources}
Typical approach:
\begin{enumerate}
    \item Define security principals: sec principal = app (unlike Linux: user)
    \item Implement reference monitor: two places: 1) in App framework for system APIs 2) part of OS for IPC calls to other apps.
    \item Define security policy: Primarily apps request permissions, access to some resources granted automatically.
\end{enumerate}{}

\paragraph{Permissions}
\begin{itemize}
    \item 4 Categories: Normal, Dangerous, Signature, SignatureOrSystem.
    \item Permissions are declared in the manifest.
    \item Problem few uers can associate privacy risks with respective permissions and also habituation. Also many developers request unnecessary permissions.
\end{itemize}{}

\paragraph{Android Malware}
\begin{itemize}
    \item Often implemented by repacking popular apps
    \item Detection: dist channels like Google Play.
    \item Detection Techniques:
    \begin{itemize}
        \item static analysis: look for known malicious code.
        \item Dynamic analysis: execute in simulated environment. (But malware can detect simulation and very expensive to explore all execution paths)
    \end{itemize}{}
    \item Leverage the fact that Android malware is often distributed by repackaging many popular apps. Take diffs from related Apps and identify suspicious code segments.
\end{itemize}{}

\subsection{Mobile Hardware Security}
\paragraph{Why hardware security?}
\begin{itemize}
    \item Ensure that correct OS security framework is run on the device requires Platform Integrity. (Secure Boot, Authenticated Boot)
    \item Run payment app on smartphone, requires isolated execution, secure storage and remote attestation or device authentication for provisioning of payment credential. (recall SXG enclaves and their trust model)
\end{itemize}{}

\paragraph{Trust anchors:}
Minimal hardware elements and functionalities needed to implement the above security requirements.\\
TODO look at slides

\paragraph{TEE (ARM TrustZone)}