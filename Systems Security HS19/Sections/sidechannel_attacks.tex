\section{Side Channel Attacks}

Security proofs are based on \textbf{models of system and attacker}. Models do not take into account the implementation and that it interacts with the environment. E.g. observing side information leakage.\\

\subsection{Classic timing attack on RSA}
    \begin{itemize}
        \item [-]leverage execution time to recover the key.
        \item [-]Applies to many asymmetric cryptosystems: RSA, EL Gamal, Diffie-Hellman.
        \item [-]Cryptosystem vulnerable, when: 
        \begin{itemize}
            \item[1.] its exec time depends on the key
            \item[2.] exec time measurable for different inputs.
        \end{itemize}
        \item [-]look at side channel attack (sl.21ff)
        \item [-]\textbf{Hamming weight:} detect how many "1"s in key - Greatly reduces search space             but still huge. Due to Key-dependant Branching in code:
        \item [-]In Montgomery multiplication the exec time depends on key and input (intermediary           value).
        \item [-] Attack main idea: Try large nr of different messages and leverage avg execution            time (with statistical test).
        \item [-] TODO: look at Sidechannel attack on slides 25.ff.
    \end{itemize}    

\subsubsection{Protecting against Timing Attacks}
\begin{itemize}
    \item[-] Point defenses are easy (make the last reduction in montgomery mult unconditional)
    \item[-] Generic Protection is hard (ensure all ops take sam time for all inputs.
    \item[-] Challenge: Performance penalty can be very high, plus difficult to implement.
\end{itemize}

\textbf{Masking by Kocher}
\begin{itemize}
    \item [-]Choose different random X for each message M
    $$SIGN(m) = [(m*X^d mod n] * [(X^-1 )^d mod n] mod n = m^d mod n$$
    \item[-] $(X^-1)^d mod n$ can be computed in advance. Two additional multiplications
\end{itemize}{}

\subsection{Cache attack on SGX enclaves (irrelevant)}
Let us now consider a collocated adversary (e.g. one VM in cloud platform attacks another VM). Attacker and VM share resources like caches, DRAM, disk.\\
\\
\subsubsection{Prime and Probe attack:}
\begin{itemize}
    \item[-] Variant1: Data access depends on secret data (e.g. read var x or write var y based on key). --> Data access patterns in the cache leak information about the secret. 
    \item[-] Variant2: Control flow depends on secret data (e.g. call function x() of y() base on key). --> Code access patterns in the cache leak information about the secret.
    \item[-] TODO look at slide 36.
\end{itemize}

\subsubsection{Protection against cache attacks}
\begin{itemize}
    \item[-]Try to eliminate secret-dependent cache accesss patterns and control flow branches.
    \item[-]E.g. Hardenet AES: when reading from one S-box touch each of them.
\end{itemize}

\subsection{Untrusted OS (irrelevant)}

Modern proessors include security architectures that attempt to protect applications from untrusted OS e.g.: Intels Software Guard Extensions \textbf{(SGX)} (enclave = security critical part of application). Outsourcing data and computation to untrusted cloud is such a scenario.\\
\\
\textbf{Background on SGX:} OS cannot directly access enclave memory.
\begin{itemize}
    \item[-] Access control checks when enclave data processed inside the CPU.
    \item[-] Data encrypted when it leaves the CPU.
\end{itemize}{}
\\
\textbf{Attacking SGX enclaves: }
\begin{itemize}
    \item[-]Challenges: Speed, Noise, Attack detection, Enclave Hardening.
    \item[-]Techniques: Performance Monitoring COunters(PMC), isolate victim, Hyper-threading, Attack target.
\end{itemize}
\\
\subsection{Cache-timing attacks on AES}

\begin{itemize}
    \item\textbf{Vulnerability: }Software implementations using look-up tables to perform internal operations of the cipher, such as S-boxes. Include both the reference and optimized implementations of AES.
    \item TODO: look at AES reference and optimized algorithm.
    \item\textbf{Basic Idea: }The time needed to access a specific position in the lookup table depends on the address of this position. Attacker compares access time to same setup locally:
    \begin{itemize}
        \item[1.]For each byte of key, one index value will ave the slowest lookup. $\rightarrow$ Total lookup will be slow. Find this message byte n that is slowest for constant random rest of message. 
        \item[2.]Find locally the slowest index i with same setup. 
        \item[3.]$K[0] \bigoplus n[0] = i$ (modular keysize K).
    \end{itemize}{}
    \item\textbf{Countermeasures: }
    \begin{itemize}
        \item[-]Avoid memory access.
        \item[-]Alternative lookup tables (different AES formulations).
        \item[-]Data-oblivious memory access pattern.
        \item[-]Hiding the timing.
        \item[$\rightarrow$]Most modern CPUS have special AES hardware that does not use RAM! $\rightarrow$ not vulnerable.
    \end{itemize}{}
\end{itemize}{}

\subsection{Power Analysis Attacks: }
An attack that enables to retrieve a secret key by observing power traces of a device.

\begin{itemize}
    \item\textbf{Types: }
    \begin{itemize}
        \item\textbf{Simple Power Analysis SPA: }Observation and analysis of power consumption during a
        single execution (trace); trace depends on the key 
        \item\textbf{Differential Power Analysis: }Statistical analysis of multiple measurements based on the
        similar principles as timing cryptanalysis; trace depends on the key and on the input plaintext/ciphertext.
        \item\textbf{High Order Differential Power Analysis: }Complex statistical analysis of multiple measurements
    \end{itemize}{}
    \item\textbf{Devices: }Mainly used on Smartcards, RFID chips, Sensor Nodes. And the Attacker needs physical access.Needs additional equipment like Oscilloscope.
    \item\textbf{SPA-Main Idea: }
    \begin{itemize}
        \item[1.]Measure power consumption of instruction sequences that depend on the key (square vs. multiply in RSA).
        \item[2.]The key is 1 where there is a multiply after square.
    \end{itemize}{}
    \item\textbf{DPA-Main Idea: }Power consumption depends not only on the type of the executed instructions but also on values of operands (unlike in the simple power analysis)
    \item\textbf{Protection: }
    \begin{itemize}
        \item[-]Desynchronization: Random injection of dummy instructions.
        \item[-]Noise generator: but then correlations can still be exploited using larger nr of measurements.
        \item[-]Software balancing: significant reduction of speed.
        \item[-]Shamir's countermeasure: add 2 capacitors s.t. one at a time is charging an one discharging. Power consumption doesn't directly depend on calculations. 
    \end{itemize}{}
\end{itemize}{}

\subsection{Accoustic attack on RSA: }
\begin{itemize}
    \item[]High frequency sounds caused by vibration of
    electronic components (capacitors and coils) in the
    computer’s voltage regulation circuit
    \item[]Different secret keys cause different operations with different power levels (as seen before)
    \item[]Voltage Regulator changes behavior to stabilize voltage
    \item[]causes different vibrations $\rightarrow$ emitted as different sounds
    \item[]Can’t detect individual CPU operations. But modular exponentiation takes more time $\rightarrow$ Try to find longer patterns
\end{itemize}

\subsection{TEMPEST: Transmitted Electro-Magnetic Pulse / Energy Standards & Testing}

\begin{itemize}
    \item[-]Compromising emanations may be generated by any electrical information generating or processing equipment.
    \item[-]Can be sensed and transmitted over air, water, electrical lines,
\end{itemize}

\subsection{Tamper-Resilience}

\begin{itemize}
    \item[-]\textbf{Tamper resistance: }systems take the bank vault approach. (e.g. Smartcards)
    \item[-]\textbf{Tamper responding: }systems use the burglar alarm approach.
    \item[-]\textbf{Tamper evident: }systems are designed to ensure that if a break-in occurs, evidence of the break-in is left behind.
\end{itemize}

\textbf{Specialized Devices}
\begin{itemize}
    \item[-]Smartcards: 
    \begin{itemize}
        \item hold secret keys
        \item perform crypto-operations (dedicated)
        \item access-protection with a PIN
        \item Limited tamper resistance
    \end{itemize}
    \item[-]Cryptographical Co-processors:
    \begin{itemize}
        \item hold secret keys
        \item perform crypto-operations (broader set)
        \item TPM functionality
        \item access protected with a (master) key
        \item Tamper resistance
    \end{itemize}{}
    
\end{itemize}{}

\subsubsection{Smartcards}

\paragraph{Invasive Attacks}

\paragraph{Semi-Invasive Attacks}
\begin{itemize}
    \item[-]\textbf{Glitch-Attack: }In a glitch attack, attacker deliberately generate a malfunction that
    causes one or more flipflops to adopt the wrong state.
    Hardware and Software countermeasures are taken.
\end{itemize}{}

\subsubsection{HSM Hardware Security Module}
\paragraph{Typical Attacks on HSM:}
\begin{itemize}
    \item[-]Cryptanalysis: the attacker exploits design flaws in the crypto primitives such
    as encryption algorithms, hash functions, and digital signature schemes (cryptanalysis techniques)
    \item[-]Protocol analysis: flaws in the protocols in which crypto primitives are used (numerous formal analysis techniques)\\
    API attacks (introduced by Andersson), extend protocol
    analysis to APIs
\end{itemize}{}

\paragraph{Security API}
\begin{itemize}
    \item[-]the top-level software component of a
    cryptoprocessor, which governs its interaction with the outside world.
    \item[-]It extends a cryptographic API by enforcing policy on the interactions as well as providing cryptographic services.
\end{itemize}{}

\paragraph{API-Attack: }

\paragraph{ISO-0 Attack:}

\paragraph{Some Solutions: }
\begin{itemize}
    \item[-]Access Control
    \item[-]Limit functionality: Enable only what an entity
    requires is allow to access. Disable all other access to
    data and functions.
    \item[-]Formal analysis techniques similar to the ones use for protocols.
\end{itemize}{}

\paragraph{Doesn't matter how secure the device is physically if it leaks secrets due to API attacks}

\paragraph{Chip and Pin is broken: $\rightarrow$ read paper}

