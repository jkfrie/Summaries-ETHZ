\section{Network verification}
\subsection{Introduction}
\begin{itemize}
    \item Each router:\begin{itemize}
        \item Runs multiple routing protocols (e.g. OSPF, BGP, ISIS)
        \item Stores a configuration (parameters used by the protocols)
    \end{itemize}
    \item Forwarding table:\begin{itemize}
        \item Defines the next hop for any packet
        \item Collectively, the forwarding tables of all routers define the forwarding plane
    \end{itemize}
\end{itemize}
\begin{minipage}{\linewidth}
    \centering      
    \def\svgwidth{\linewidth}
    \input{Figures/L8_networkConfiguration}    
\end{minipage}
\paragraph{Why network configuration is hard:\newline}
We have multiple protocols (BGP, OSPF, IS-IS), each with their own low-level configurations (BGP-Policies, Link costs, Static routes). Also the protocols themselves interact with each other (Preferences like: static over OSPF; Dependencies: BGP relies OSPF)\newline
\paragraph{Example\newline}
\begin{minipage}{\linewidth}
    \centering      
    \def\svgwidth{\linewidth}
    \input{Figures/L8_networkConfiguration2}    
\end{minipage}
\begin{minipage}{\linewidth}
    \centering      
    \def\svgwidth{\linewidth}
    \input{Figures/L8_ExampleExtended}    
\end{minipage}
\subsection{Analysis of Network Configurations via Datalog}
\begin{minipage}{\linewidth}
    \centering      
    \def\svgwidth{\linewidth}
    \input{Figures/L8_networkAnalysisWithDatalog}    
\end{minipage}
\subsection{Batfish}
\begin{itemize}
\item Network configuration analyzer
 \item Available at http://www.batfish.org/
 \item Has found real bugs in real networks
 \item Three steps:
 \begin{enumerate}
    \item Parse configurations (to derive input facts)
    \item Compute forwarding plane (by computing a fixedpoint)
    \item Check for violations (by querying the fixed-point)
\end{enumerate}
\end{itemize}

\paragraph{Step 1: Parse configurations\newline}
\begin{minipage}{\linewidth}
    \centering      
    \def\svgwidth{\linewidth}
    \input{Figures/L8_configParsing}    
\end{minipage}
\paragraph{Step 2: Compute forwarding plane}
\begin{itemize}
    \item Compute OSPF routes
    \begin{itemize}
    \item Input: OSPFLink(Router1, Router2, Cost)
    \item Wanted Output: BestOSPFRoute(Router, Network, NextHop)\newline
    \begin{minipage}{\linewidth}
    \centering      
    \def\svgwidth{\linewidth}
    \input{Figures/L8_step2_OSPF}    
\end{minipage}
\end{itemize}
    \item Compose protocols (by given preference)
    \begin{itemize}
        \item Inputs: 
        \begin{itemize}
            \item bestOSPFRoute(Router, Network, NextHop, Cost)
            \item static(Router, Network, NextHop)
            \item localPref(Router, Protocol, Preference)
        \end{itemize}
        \item Output: fwd(Router, Network, NextHop)\newline
        \begin{minipage}{\linewidth}
        \centering      
        \def\svgwidth{\linewidth}
        
        \input{Figures/L8_step2_fwd}   
        \end{minipage}
    \end{itemize}
\end{itemize}
\paragraph{Step 3: Analyze forwarding plane\newline}
For example: Detect Multi-path Inconsistencies with the following\newline
\vspace{0.5cm}
\begin{minipage}{0.5\linewidth}
        \centering      
        \def\svgwidth{\linewidth}
        \input{Figures/L8_multipath}   
\end{minipage}
\vspace{0.5cm}
\newline More properties:\newline
\begin{itemize}
    \item Paths: Packets for traffic class tc must follow the path \texttt{r1 -> ... -> r10}\newline \texttt{holds <- fwd(r1, tc, r2), ..., fwd(r9, tc, r10)}
    \item Traffic isolation: The paths for two distinct traffic classes \texttt{tc1} and \texttt{tc2} do not share links in the same direction\newline 
\texttt{holds <- $\forall$ R1, R2. fwd(R1, tc1, R2) => (!fwd(R1, tc2, R2))}
    \item Reachability: Router can reach a given traffic class
    \item Loop-freeness: The forwarding plane has no loops
\end{itemize}


